\section{RMSE calculation code}
\lstset{language=C++}
\begin{lstlisting}
#include <iostream>
#include <iomanip>
using namespace std;

#include "window_fn_top.h"

#ifdef FLOAT_DATA
#define ABS_ERR_THRESH 0.0
#else
#define ABS_ERR_THRESH 0.001
#endif

#define WINDOW_FN_DEBUG 1

int main(void)
{
   win_fn_in_t testdata[WIN_LEN];
   win_fn_out_t hw_result[WIN_LEN];
   float sw_result[WIN_LEN];
   unsigned err_cnt = 0;

   // Generate test vectors and expected (real valued) results
   for (int i = 0; i < WIN_LEN; i++) {
      float samp_i = 64.0;
      // Use the coefficient calculation function from class for convenience
      float win_i = xhls_window_fn::coef_calc<WIN_LEN,WIN_TYPE>(i);
      testdata[i] = samp_i; // implicit conversion from float to ap_fixed<>
      sw_result[i] = samp_i * win_i; // generate floating point expected values
   }

   // Run the DUT
   window_fn_top(hw_result, testdata);

   // Check results
   cout << "Checking results against a tolerance of " << ABS_ERR_THRESH << endl;
   cout << fixed << setprecision(5);
   float sum = 0;
   int count = 0;
   for (unsigned i = 0; i < WIN_LEN; i++) {
      float abs_err = float(hw_result[i]) - sw_result[i];
      sum += abs_err * abs_err;
      count ++;

#if WINDOW_FN_DEBUG
      cout << "i = " << i << "\thw_result = " << hw_result[i];
      cout << "\t sw_result = " << sw_result[i] << endl;
#endif
      if (fabs(abs_err) > ABS_ERR_THRESH) {
         cout << "Error threshold exceeded: i = " << i;
         cout << "  Expected: " << sw_result[i];
         cout << "  Got: " << hw_result[i];
         cout << "  Delta: " << abs_err << endl;


         err_cnt++;
      }
   }
   cout << endl;

   // Print final status message
   if (err_cnt) {
      cout << "!!! TEST FAILED - " << err_cnt;
      cout << " results out of tolerance." << endl;
   } else{
      cout << "Test Passed" << endl;
   }
   // Only return 0 on success
   cout << "RMSE " << sqrt(sum/count) << endl;
   return err_cnt;
}


\end{lstlisting}

\section{Dot-Product calculation code}
\lstset{language=C++}
\begin{lstlisting}


#include "fir.h"

void dotProduct(
    data_t x[N],
    data_t y[N],
    data_t *output
) {
    data_t acc = 0;
    int i;
    Multloop: for(i=0; i < N; i++) {
#pragma HLS UNROLL factor=20
        acc += y[i]*x[i];
    }
    *output = acc;
}


\end{lstlisting}

\section{Dot-Product test bench}
\lstset{language=C++}
\begin{lstlisting}

#include <stdio.h>
#include <math.h>
#include "fir.h"

int main () {
  FILE         *fp;

  data_t signal, output;
  data_t progout = 0;
  data_t x[N] = {5, 7, -3, -4};
  data_t y[N] = {-2, 3, -5, 6};

  int i;
  
  fp=fopen("out.dat","w");

  dotProduct(x,y,&output);

  for (i=0;i<=N;i++) {
      progout = progout + (x[i]*y[i]);
  }

  fprintf(fp, "%i %i\n", progout, output);
  fclose(fp);
  
  printf ("Comparing against output data \n");
  if (system("diff -w out.dat out.gold.dat")) {

    fprintf(stdout, "*******************************************\n");
    fprintf(stdout, "FAIL: Output DOES NOT match the golden output\n");
    fprintf(stdout, "*******************************************\n");
     return 1;
  } else {
    fprintf(stdout, "*******************************************\n");
    fprintf(stdout, "PASS: The output matches the golden output!\n");
    fprintf(stdout, "*******************************************\n");
     return 0;
  }
}

\end{lstlisting}

\section{Python binary counter}
\lstset{language=Python}
\begin{lstlisting}
from time import sleep
from pynq.overlays.base import BaseOverlay

base = BaseOverlay("base.bit")

Delay1 = 0.3
Delay2 = 0.1
color = 0
rgbled_position = [4,5]

for i in range(7):
    for led in rgbled_position:
        base.rgbleds[led].write(i)
        base.rgbleds[led].write(i)
    sleep(Delay2)

for led in base.leds:
    led.on()    
while (base.buttons[3].read()==0):
    if (base.buttons[0].read()==1):
        color = (color+1) % 8
        for led in rgbled_position:
            base.rgbleds[led].write(color)
            base.rgbleds[led].write(color)
        sleep(Delay1)
        
    elif (base.buttons[1].read()==1): # LEDS Count-Up
        for n in range(16):
            for j in range(4):
                base.leds[j].off()
                
            if (n >= 8):
                base.leds[3].on()
                n -= 8
            if (n >= 4):
                base.leds[2].on()
                n -= 4
            if (n >= 2):
                base.leds[1].on()
                n -= 2
            if (n ==1):
                base.leds[0].on()
            sleep(Delay1)
            
        for j in range(4):
                base.leds[j].off()
            
    elif (base.buttons[2].read()==1):
        for led in reversed(base.leds):
            led.off()
        sleep(Delay2)
        for led in reversed(base.leds):
            led.toggle()
            sleep(Delay2)                  
    
print('End of this demo ...')
for led in base.leds:
    led.off()
for led in rgbled_position:
    base.rgbleds[led].off()
\end{lstlisting}

\section{Horizontal Edge Detection}
\lstset{language=C++}
\begin{lstlisting}
#include <ap_fixed.h>
#include <ap_int.h>
#include <stdint.h>
#include <assert.h>

typedef ap_uint<8> pixel_type;
typedef ap_int<8> pixel_type_s;
typedef ap_uint<96> u96b;
typedef ap_uint<32> word_32;
typedef ap_ufixed<8,0, AP_RND, AP_SAT> comp_type;
typedef ap_fixed<10,2, AP_RND, AP_SAT> coeff_type;

struct pixel_data {
    pixel_type blue;
    pixel_type green;
    pixel_type red;
};


void template_filter(volatile uint32_t* in_data, volatile uint32_t* out_data, int w, int h, int parameter_1){
#pragma HLS INTERFACE s_axilite port=return
#pragma HLS INTERFACE s_axilite port=parameter_1
#pragma HLS INTERFACE s_axilite port=w
#pragma HLS INTERFACE s_axilite port=h

#pragma HLS INTERFACE m_axi depth=2073600 port=in_data offset=slave // This will NOT work for resolutions higher than 1080p
#pragma HLS INTERFACE m_axi depth=2073600 port=out_data offset=slave


    int row [1920] = {0}; 

    for (int y = 0; y < h; ++y) {

        for (int x = 0; x < w; ++x) {

            #pragma HLS PIPELINE II=1
            #pragma HLS LOOP_FLATTEN off

            unsigned int current = *in_data++;

            unsigned char in_r = current & 0xFF;
            unsigned char in_g = (current >> 8) & 0xFF;
            unsigned char in_b = (current >> 16) & 0xFF;

            unsigned char out_r = 0;
            unsigned char out_b = 0;
            unsigned char out_g = 0;

            if (y>h/2){
                float curr_gray = 0.2126f*in_r  + 0.7152f*in_g  + 0.0722f*in_b ;
                int Y= abs(int(curr_gray)-row[x]);
                row[x] = curr_gray;

                out_r = Y;
                out_b = Y;
                out_g = Y;

            }
            else{
                out_r = in_r;
                out_g = in_g;
                out_b = in_b;
            }

            unsigned int output = out_r | (out_g << 8) | (out_b << 16);
            *out_data++ = output;

        }

    }

}

\end{lstlisting}

\section{Javascript code for simple colour detection using the p5.js library.}
\lstset{language=Java}
\begin{lstlisting}

var video, x, y, r, g, b, index, wr,wg,wb, slider;

function setup() {
 createCanvas(640, 480)
 pixelDensity(1);
 video = createCapture(VIDEO);
 video.size(width, height);
 slider = createSlider(0,1,0.1,0.05);
 slider.position(width/1.4,height/1.2);
 slider.style('width', '160px');

 mode = 1;
 
 wr = 1;
 wg = 223/255;
 wb = 196/255;//255,223,196
}


function draw() {
 video.loadPixels();
 //image(video, 0, 0);
 background(0);
 var val = slider.value();
 find(val);
}

function mouseReleased() {
  index = (mouseX + 1 + (mouseY * video.width)) * 4;
  wr = 1;
  wg = video.pixels[index + 1]/video.pixels[index + 0];
  wb = video.pixels[index + 2]/video.pixels[index + 0];
 console.log({wr,wg,wb});
}

function find(approx) {
 var tx = 0;
 var ty = 0;
 var counter = 0;
 for(y=0;y<video.height;y++) {
  for(x=0;x<video.width;x++) {
   index = (x + 1 + (y * video.width)) * 4;
   r = video.pixels[index + 0];
   g = video.pixels[index + 1];
   b = video.pixels[index + 2];
   
   if(abs(g/r-wg)<approx && abs(b/r-wb)<approx) {
    tx += x;
    ty += y;
    stroke(255)
    point(x,y);
    counter++;
   }
  }
 }
 tx /= counter;
 ty /= counter;
 noStroke();
 fill(255,0,0);
 //rect(tx,ty,50,50);
}

\end{lstlisting}